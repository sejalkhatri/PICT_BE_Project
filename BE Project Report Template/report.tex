%This is a very basic  BE PROJECT PRELIMINARY template.

%############################################# 
%#########Author :  PROJECT###########
%#########COMPUTER ENGINEERING############


\documentclass[oneside,a4paper,12pt]{report}
%\usepackage{showframe}
%\hoffset = 8.9436619718309859154929577464789pt
%\voffset = 13.028169014084507042253521126761pt

\fancypagestyle{plain}{%
  \fancyhf{}
  \fancyfoot[CE]{College_Name, Department of Computer Engineering 2015}
  \fancyfoot[RE]{\thepage}
}
\pagestyle{fancy}
\fancyhead{}
\renewcommand{\headrulewidth}{0pt}
\footskip = 0.625in
\cfoot{}
\rfoot{}

\usepackage[]{hyperref}
\usepackage{tikz}
\usetikzlibrary{arrows,shapes,snakes,automata,backgrounds,petri}

\usepackage{tabularx}

\usepackage[nottoc,notlot,notlof,numbib]{tocbibind}
\usepackage[titletoc]{appendix}
\usepackage{titletoc}
\renewcommand{\appendixname}{Annexure}
\renewcommand{\bibname}{References}

\setcounter{secnumdepth}{5}

\usepackage{float}
\usepackage{subcaption}
\usepackage{multirow}

%\usepackage[ruled,vlined]{algorithm2e}

\begin{document}

\setlength{\parindent}{0mm}
\begin{center}
{\bfseries SAVITRIBAI PHULE PUNE UNIVERSITY \\}
 \vspace*{1\baselineskip}
{\bfseries A PRELIMINARY PROJECT REPORT ON \\}
 \vspace*{2\baselineskip}
{\bfseries \fontsize{16}{12} \selectfont Physical Web with Vending Machine \\ \vspace*{2\baselineskip}}
{\fontsize{12}{12} \selectfont SUBMITTED TOWARDS THE
 \\PARTIAL FULFILLMENT OF THE REQUIREMENTS OF \\

\vspace*{2\baselineskip}}
{\bfseries \fontsize{14}{12} \selectfont BACHELOR OF ENGINEERING (Computer
Engineering) \\
\vspace*{1\baselineskip}} 
{\bfseries \fontsize{14}{12} \selectfont BY \\ 
\vspace*{1\baselineskip}} 
Student Name: Sejal Khatri \hspace{25 mm} Exam No: B120054336\\
Student Name: Amruta Ranade \hspace{25 mm} Exam No:  B120054223  \\
Student Name: Kevin Kaul\hspace{25 mm} Exam No: B120054333  \\
\vspace*{2\baselineskip}
{\bfseries \fontsize{14}{12} \selectfont Under The Guidance of \\  
\vspace*{2\baselineskip}} 
Prof. A.R.Deshpande\\
\includegraphics[width=100pt]{collegelogo.png} \\
{\bfseries \fontsize{14}{12} \selectfont DEPARTMENT OF COMPUTER ENGINEERING \\
Pune Institute of Computer Technology \\
Dhankawadi,Pune-411043.
}
\end{center}

\newpage



\begin{figure}[ht]
\centering
\includegraphics[width=100pt]{collegelogo.png}
\end{figure}


{\bfseries \fontsize{14}{12} \selectfont \centerline{Pune Institute of Computer Technology}
\centerline{DEPARTMENT OF COMPUTER ENGINEERING}
\vspace*{3\baselineskip}} 


{\bfseries \fontsize{16}{12} \selectfont \centerline{CERTIFICATE} 
\vspace*{3\baselineskip}} 

\centerline{This is to certify that the Project Entitled}
\vspace*{1\baselineskip} 


{\bfseries \fontsize{14}{12} \selectfont \centerline{ Physical Web with Vending machine.}
\vspace*{1\baselineskip}}

\centerline{Submitted by}
\vspace*{1\baselineskip} 
\centerline{Sejal Khatri  \hspace{25 mm} Exam No:B120054336} 
\centerline{Amruta Ranade \hspace{25 mm} Exam No:B120054223  } 
\centerline{Kevin Kaul \hspace{25 mm} Exam No:B120054333 }
\vspace*{1\baselineskip} 
is a bonafide work carried out by Students under the supervision of Prof.A.R.Deshpande and it
is submitted towards the partial fulfillment of the requirement of Bachelor of Engineering (Computer Engineering) Project.\\\\\\

\bgroup
\def\arraystretch{0.7}
\begin{tabular}{c c }
Prof. A.R.Deshpande &  \hspace{50 mm} Prof. Rajesh Ingle \\								
Internal Guide   &  \hspace{50 mm} H.O.D \\
Dept. of Computer Engg.  &	\hspace{50 mm}Dept. of Computer Engg.  \\
\end{tabular}
%}



\newpage

%\pictcertificate{TITLE OF BE PROJECT}{Student Name}{Exam Seat No}{Guide Name}
\setcounter{page}{0}
\frontmatter
\cfoot{PICT, Department of Computer Engineering 2016}
\rfoot{\thepage}
\pagenumbering{Roman}
%\pictack{Physical Web with Vending Machine}{Prof.A.R.Deshpande}

		
{  \newpage {\bfseries \fontsize{14}{12} \selectfont \centerline{Abstract} 
\vspace*{2\baselineskip}} \setlength{\parindent}{11mm} }
{ \setlength{\parindent}{0mm} }
 A vending machine is a machine that dispenses items such as snacks, bever-
ages to customers automatically, after the customer inserts currency or credit
into the machine. But nowadays paying in cash has become a difficulty and cannot be fulfilled every time.Therefore we provide a platform for the vending machine functionalities and management to be handled by cloud using Internet of things.With this approach Online payment for vending machines can be made possible and the
stock record is maintained on the cloud for dynamically updating the vendor.
In addition to this the users are notified about the presence of the vending
machine using Web Bluetooth API.


{  \newpage {\bfseries \fontsize{14}{12} \selectfont \centerline{Acknowledgments} 
\vspace*{2\baselineskip}} \setlength{\parindent}{11mm} }
{ \setlength{\parindent}{0mm} }

\textit{It gives us great pleasure in presenting the preliminary project report 
on {\bfseries \fontsize{12}{12} \selectfont `Physical Web with Vending machine'}.}
\vspace*{1.5\baselineskip}

 \textit{We would like to take this opportunity to thank our internal guide
 \textbf{Prof. A.R.Deshpande} for giving us all the help and guidance we needed. We are really grateful to her for her kind support. Her valuable suggestions were very helpful.} \vspace*{1.5\baselineskip}

 \textit{We are also grateful to \textbf{Prof. Rajesh Ingle}, Head of Computer
 Engineering Department, PICT for his indispensable
 support and suggestions.}
\vspace*{1.5\baselineskip}

\textit{In the end our special thanks to our external guide \textbf{Mr. Anuj Deshpande} for
providing various resources such as  laboratory with all needed software platforms,
continuous Internet connection, for Our Project.}
\vspace*{3\baselineskip} \\
\begin{tabular}{p{8.2cm}c}
&Sejal Khatri\\
&Amruta Ranade\\
&Kevin Kaul\\
&(B.E. Computer Engg.)
%}
\end{tabular}


% \maketitle
\tableofcontents
\listoffigures 
\listoftables



\mainmatter



  \titleformat{\chapter}[display]
{\fontsize{16}{15}\filcenter}
{\vspace*{\fill}
 \bfseries\LARGE\MakeUppercase{\chaptertitlename}~\thechapter}
{1pc}
{\bfseries\LARGE\MakeUppercase}
[\thispagestyle{empty}\vspace*{\fill}\newpage]







\setlength{\parindent}{11mm}
\chapter{Synopsis}

\section{Project Title}
Physical Web with Vending Machine
\section{ Project Option }
Industry sponsored

\section{Internal Guide}
Prof. A.R.Deshpande

\section{ Sponsorship and External Guide} 
Sponsored By : Marvell Pvt.ltd.

\section{Technical Keywords (As per ACM Keywords)}
% {\bfseries Technical Key Words:}      
% \begin{itemize}
%   \item 	Cloud Computing
% \item	Service Composition
% \item	Online Web services
% \end{itemize}
IOT, Cloud Computing, Cloud based storage, Web Application, Web Services, Web based interaction, Web Interfaces.

\section{Problem Statement}
\label{sec:problem}
        To Automate vending machine functionalities for vendors and enabling easy accessibility for users through online payment and establishing a physical interface with the help of beacons. 
\section{Abstract}
A vending machine is a machine that dispenses items such as snacks, bever-
ages to customers automatically, after the customer inserts currency or credit
into the machine. But nowadays paying in cash has become difficult and cannot be fulfilled every time. Therefore we provide a platform for the vending machine functionalities and management to be handled by cloud using Internet of things.With this approach online payment for vending machines can be made possible and the
stock record is maintained on the cloud for dynamically updating the vendor.
In addition to this the users are notified about the presence of the vending
machine in nearby area  using Web Bluetooth API.
		    		   
\section{Goals and Objectives}
Project Goal :Presently operating the vending machines is not user-friendly and it is observed to be time consuming as well.Our project goal is to increase the scope and quality of the vending machine services provided to the people. 
\begin{enumerate}

\item Project Objective 1:
Set up beacons on the vending machine.
People use coins or paper money while operating the vending machines due to which there arises a problem when the user does not seem to have exact change with him.
Performance Measure :
Online (cashless) payments are made available for the users for easy purchase of items.
\item Project Objective 2:
The vendors are not aware about the stock required in the vending machines when excess usage of the products occur.
Performance Measure :
Vendors are well informed about the stock management of the machine and are also aware of the customers past transactions.
\item Project Objective 3 :
Finding a vending machine in new locations everytime becomes difficult for people.
Performance Measure :
The users are notified about the presence of the vending machine available in their current location using the Web Bluetooth API.

\end{enumerate}	
\section{Relevant mathematics associated with the Project}
\label{sec:math}
System Description:\\
 Let S be the solution system ,
	  S = {s , e, X, Y, F, DD, NDD , sc, fc | shmem}\\
            where,    \\   
s = start state {Wi-fi interfacing}\\
e = end state { Product delivered and status recorded}\\
X = Input set\\
Y = Output set\\
Input:(Physical address, user’s choice)	\\ 
Output:(Product requested , suggestions)\\ 
Functions : Fme + Ffriend.\\
Fme = Main functions.\\
Fme = (fin , fout,initiate,detect,connect).\\
Fin : {Faddress , Fchoice}\\
Fout:{Fdispose , Fsuggest}\\
\begin{center}
	\begin{figure}[!htbp]
		\centering
		\fbox{\includegraphics[height=150pt]{function.png}}
	  \caption{Function Diagram}
	  \label{fig:act-dig}
	\end{figure}
\end{center}  
Finitiate :{Fconnectwifi ,Fflashurl,Fconnectaws}\\
Fdetect:{Fdetectwifi}\\
Fconnect:{Fcw ,Fcaws}\\
Fcw :refresh connection wifi\\
Fcaws :refresh connection AWS\\
Ffriend = inbuilt functions.\\
Ffriend = (fproc , fcloud).\\
Non Deterministic Data : Physical address of user's device and location.\\
Deterministic Data :Items(as flashed on users device)\\
Success Conditions:Valid input( i.e valid user choice) is given and the product is dispensed successfully and proper internet availability.\\
Failure Conditions: Invalid input( i.e invalid user choice)  given and product not dispensed along with poor internet availability.\\


\section{Names of Conferences / Journals where papers can be published}

Journal of Internet Services and Applications\\
IEEE Cloud Computing\\
IEEE International Conference on Communications.\\
International Journal of Advanced Computing and Technology\\


\section{Review of Conference/Journal Papers supporting Project idea}
\label{sec:survey}
\begin{enumerate}

\item Zaruba, G.v., S. Basagni, and I. Chlamtac. "Bluetrees-scatternet Formation to Enable Bluetooth-based Ad Hoc Networks." ICC 2001. IEEE International Conference on Communications. Conference Record (Cat. No.01CH37240) (n.d.): n. pag. Web\\
Bluetooth is an open specification for short-range
wireless communication and networking, mainly intended to be
a cable replacement between portable and/or fixed electronic de-
vices. The specification also defines techniques for interconnecting
large number of nodes in scatternets, thus enabling the establish-
ment of a mobile ad hoc network (MANET). While several solutions
and commercial products have been introduced for one-hop Blue-
tooth communication, the problem of scatternet formation has not
yet been dealt with. This problem concerns the assignment of the
roles of master and slave to each node so that the resulting MANET
is connected. In this paper they introduce two novel protocols for
forming connected scatternets. In both cases, the resulting topol-
ogy is termed a bluetree. In our bluetrees the number of roles each
node can assume are limited to two or three (depending on the
protocol), thus imposing low slave management overhead. The ef-
fectiveness of both protocols in forming MANETs is demonstrated
through extensive simulations.\\
\item Linthicum, David S. "The Technical Case for Mixing Cloud Computing and Manufacturing." IEEE Cloud Computing 3.4 (2016): 12-15. Web.\\
Today’s manufacturing processes often lack visibility into resource consumption metrics, productivity, and even logistics. The end result is huge blind spots in the building process that often mean a lack of productivity and efficiency, leading manufacturing companies down unprofitable paths.\\
\item Massuthe, P., and K. Schmidt. "Operating Guidelines - an Automata-Theoretic Foundation for the Service-Oriented Architecture." Fifth International Conference on Quality Software (QSIC'05) (n.d.): n. pag. Web.\\
In the service-oriented architecture (SOA), we distinguish three roles of service owners: service providers, service requesters, and service brokers. Each service provider publishes information to the broker about how requesters can interact with its service. Thus, the broker can assign a fitting service provider to a querying requester. We propose the information published to the broker to be operating guidelines. Operating guidelines are essentially communication instructions for the service requester.
They present an automata-theoretic approach that is centered around operating guidelines and is capable of implementing all tasks arising in the SOA.\\

\item Sneps-Sneppe, Manfred, and Dmitry Namiot. "On Physical Web Models." 2016 International Siberian Conference on Control and Communications (SIBCON) (2016): n. pag. Web.\\
The Physical Web is a generic term describes interconnection of physical objects and web. The Physical Web lets to present physical objects in a web. There are different ways
to do that and we will discuss them in our paper. Usually, the web presentation for a physical object could implement with the help of mobile devices. The basic idea behind the Physical Web is to navigate and control physical objects in the world surrounding
mobile devices with the help of web technologies. Of course, there are different ways to identify and enumerate physical objects. In this paper, they describe the existing models as well as related challenges. In our analysis, we will target objects enumeration
and navigation as well as data retrieving and programming for the Physical Web. \\
\item Namiot, Dmitry, and Manfred Sneps-Sneppe. "The Physical Web in Smart Cities." 2015 Advances in Wireless and Optical Communications (RTUWO) (2015): n. pag. Web.\\
The AWS Cloud Transformation Maturity Model (CTMM) maps the maturity of
an IT organization’s process, people, and technology capabilities as they move
through the four stages of the journey to the AWS Cloud: project, foundation,
migration, and optimization. The objective of the CTMM is to help enterprise IT
organizations understand the significant challenges they might face to adopt
AWS, learn best practices and activities to handle those challenges, and recognize
the signs of maturity or expected outcomes to gauge their maturity and readiness
at every stage. This whitepaper can guide organizations to measure their
readiness for the AWS Cloud, build an effective cloud transformation strategy,
and drive an effective execution plan.\\
\item Lee, Jin-Shyan, Yu-Wei Su, and Chung-Chou Shen. "A Comparative Study of Wireless Protocols: Bluetooth, UWB, ZigBee, and Wi-Fi." IECON 2007 - 33rd Annual Conference of the IEEE Industrial Electronics Society (2007): n. pag. Web.\\
Bluetooth (over IEEE 802.15.1), ultra-wideband(UWB, over IEEE 802.15.3), ZigBee (over IEEE 802.15.4), and Wi-Fi (over IEEE 802.11) are four protocol standards for short-range wireless communications with low power consumption. From an application point of view, Bluetooth is intended for a cordless mouse, keyboard, and hands-free headset, UWB is oriented to high-bandwidth multimedia links, ZigBee is designed for reliable wirelessly networked monitoring and control networks, while Wi-Fi is directed at computer-to-computer connections as an extension or substitution of cabled networks. In this paper, they provide a study of these popular wireless communication standards, evaluating their main features and behaviors in terms of various metrics, including the transmission time, data coding efficiency, complexity, and power consumption.It is believed that the comparison presented in this paper would benefit application engineers in selecting an appropriate protocol.\\
\item Simeone, Osvaldo, and Haim H. Permuter. "Source Coding with Delayed Side Information." 2012 IEEE International Symposium on Information Theory Proceedings (2012): n. pag. Web.\\
We study source coding in the presence of side information, when the system can take actions that affect the availability, quality, or nature of the side information. We begin by extending the Wyner-Ziv problem of source coding with decoder side information to the case where the decoder is allowed to choose actions affecting the side information. We then consider the setting where actions are taken by the encoder, based on its observation of the source. Actions may have costs that are commensurate with the quality of the side information they yield, and an overall per-symbol cost constraint may be imposed. We characterize the achievable tradeoffs between rate, distortion, and cost in some of these problem settings. Among our findings is the fact that even in the absence of a cost constraint, greedily choosing the action associated with the “best” side information is, in general, suboptimal.A few examples are worked out.\\
\item Zhang, Qi, Lu Cheng, and Raouf Boutaba. "Cloud Computing: State-of-the-art and Research Challenges." Journal of Internet Services and Applications 1.1 (2010): 7-18. Web.\\
Cloud computing has recently emerged as a new paradigm for hosting and delivering services over the Internet. Cloud computing is attractive to business owners as it eliminates the requirement for users to plan ahead for provisioning, and allows enterprises to start from the small and increase resources only when there is a rise in service demand. However, despite the fact that cloud computing offers huge opportunities to the IT industry, the development of cloud computing technology is currently at its infancy, with many issues still to be addressed. In this paper, they present a survey of cloud computing, highlighting its key concepts, architectural principles, state-of-the-art implementation as well as research challenges. The aim of this paper is to provide a better understanding of the design challenges of cloud computing and identify important research directions in this increasingly important area.\\
\end{enumerate}   

\section{Plan of Project Execution}

\begin{table}[!htbp]
\begin{center}
%\def\arraystretch{1.5}
\def\arraystretch{1.5}
\begin{tabularx}{\textwidth}{| X | X |}
\hline
Activity	& Planned months\\
\hline
Requirement gathering and feasibility studying        &1 july – 15 Aug\\
\hline
Planning Activities       &16 Aug – 31 Aug\\
\hline
Designing Modules        &1 sept – 31 Oct\\
\hline
Implementation           &1 Nov – 14 Jan\\
\hline
Testing                  &15 Jan – 15 Feb\\
\hline
Deployment               &16 Feb – 28 Feb\\
\hline



\end{tabularx}
\end{center}
\caption{Project Plan}
\label{tab:usecase}
\end{table}
\subsubsection{TimeLine Chart}
\begin{center}
	\begin{figure}[!htbp]
		\centering
		\fbox{\includegraphics[height=150pt]{gantt.png}}
	  \caption{Timeline Chart}
	  \label{fig:act-dig}
	\end{figure}
\end{center}  
\newpage


\chapter{Technical Keywords}
\section{Area of Project}
Internet of Things
\section{Technical Keywords}
% {\bfseries Technical Key Words:}      
% \begin{itemize}
%   \item 	Cloud Computing
% \item	Service Composition
% \item	Online Web services
% \end{itemize}

\begin{enumerate}
	 
\item Internet of Things.
\item Cloud Computing.

\end{enumerate}

			
\chapter{Introduction}
\section{Project Idea}

 We are proposing a platform for vending machine functionalities and management to be  handled by cloud using Internet of things.Online payment for vending machines made possible and stock record maintained on the cloud for dynamically updating the vendor .


\section{Motivation of the Project}  

 Presently the people use coins or paper money while operating the vending machines due to this there arises a problem when the user does not seem to have exact change with him and the vendors dont come to know about the stock required in the vending machines when excess usage is done.Hence to minimize these problems we provide an efficient solution.

\section{Literature Survey}
 Physical Web with Vending machine.\\
Papers referred: Refer to Paper Reveiw part given before.\\



\begin{table}[!htbp]
\begin{center}
%\def\arraystretch{1.5}
\def\arraystretch{1.5}
\begin{tabularx}{\textwidth}{| X | X | X | X | X |}
\hline
Parameters	&Paper1 &Paper2 &Paper3 &Paper4\\
\hline
Topic       &Cloud computing: State -of-the-Art and Research challenges &On Physical Web Models. &Bluetrees-scatternet Formation to Enable Bluetooth-based Ad Hoc Networks. &A Comparative Study of Wireless Protocols: Bluetooth, UWB, ZigBee, and Wi-Fi.\\
\hline

Paper Type      &Journal &Research  &Study  &Conference\\
\hline

Objective      & To present a survey of cloud computing, highlighting its key concepts, architectural principles, state-of-the-art implementation as well as research challenges.& The existing physical web models as well as related challenges are described in this paper. &Two novel protocols for forming connected scatternets are introduced. &To provide a study of these popular wireless communication standards.\\
\hline

Result &To provide a better understanding of the design challenges of cloud computing and identify important research directions in this increasingly important area. &Objects enumeration and navigation as well as data retrieving and programming for the Physical Web. &The effectiveness of both protocols in forming MANETs is demonstrated
through extensive simulations. &Study of these popular wireless communication standards, evaluating their main features and behaviors in terms of various metrics, including the transmission time, data coding efficiency, complexity, and power consumption made.\\
\hline


\end{tabularx}
\end{center}
\caption{Literature Survey}
\label{tab:usecase}
\end{table}



\chapter{Problem Definition and scope}
\section{Problem Statement}
 To automate vending machine functionalities for vendors and enabling easy accessibility for users through online payment and establishing a physical interface with the help of beacons. 


\subsection{Goals and objectives}  
Project Goal :Presently operating the vending machines is not user-friendly and it is observed to be time consuming as well.Our project goal is to increase the scope and quality of the vending machine services provided to the people. 
Project Objective 1:
People use coins or paper money while operating the vending machines due to which there arises a problem when the user does not seem to have exact change with him.
Performance Measure :
Online (cashless) payments are made available for the users for easy purchase of items.
Project Objective 2:
The vendors are not aware about the stock required in the vending machines when excess usage of the products occur.
Performance Measure :
Vendors are well informed about the stock management of the machine and are also aware of the customers past transactions.
Project Objective 3 :
Finding a vending machine in new locations everytime becomes difficult for people.
Performance Measure :
The users are notified about the presence of the vending machine available in their current location using the Web Bluetooth API.
	
 \subsection{Statement of scope} 
This project will consist of creating a platform for vending machine functionalities and management to be  handled by cloud using Internet of things.Online payment for vending machines is made possible and stock record is maintained on the cloud for dynamically updating the vendor. Modules of the platform will include a firmware where hardware is used,cloud communication and a frontend available for users as well as for vendors respectively.\\
Type of vending machine : Product base.
Products: Bisleri,  coca cola and other beverages.
Device type : Mobiles which support chrome browser (e.g.Android/ios)\\
Limit of the project will be internet dependency ,so better connection (3Mbps) is required.\\
Functionality mechanism is concentrated on removing the cash payment barrier on the vending machine .\\
Final product will be used at public places like railway stations,airports,bus stands and can be used by private vendors.\\
\section{Software context}
The project will be effectively used at public places where vending machine's are placed.\\
Places like Airports, Railway stations, Bus stands and other public places have this facility coming up to support the Digital India initiative by going cashless.\\

\section{Major Constraints}
Need of google chrome browser: The user should have google chrome browser to take advantage of this service as currently the web bluetooth technology is supported by only google chrome. \\
Availability of Wi-Fi connections: The vending machine should be placed in the place where wifi availability is there ,for smoother connection and faster product delivery.If the internet fluctuates then user will have to wait for product and this in turn lead to decrease in product sale.\\
Location of the vending machines: Vending machine should be placed where it can be accesible to the people i.e the bluetooth range.\\


\section{Methodologies of Problem solving and efficiency issues}
The single problem can be solved by different solutions.  This considers the performance parameters for each approach. Thus considers the efficiency issues.

Method1:


Method2:

\section{Scenario in which multi-core, Embedded and Distributed Computing used}

 We will have a Wi-Fi enabled board on the vending machines which will send beacons to the mobile phones having Bluetooth in their vicinity.
 
 The user will receive a notification which will contain a URL flashed by the beacon on his cellphone along with suggestions for other products.

The user will hence find and approach the vending machine and click on the desired product on his cellphone.

 The transaction is carried out using the online payment gateways or mobile wallet and is recorded and stored in the cloud which is used for future transactions.

 The vending machine then dispenses the product ordered by the user.

The vendor will have an updated list of the stock remaining in the cloud.

\section{Outcome}

 Online (cashless) payments are available for the users for easy purchase of items.
Vendors are well informed about the stock management of the machine and they are also aware of the customers past transactions.

\section{Applications}

Vending machines in airports,malls and offices.

\section{Hardware Resources Required}
\begin{enumerate}
\item Beacons.
\item Vending machines.
\item Knit board(wifi-enabled micro controller).
\item Mobile phone.(Android(Version: 6.0) or iOS(version: 8))
\end{enumerate}


\section{Software Resources Required}
Platform : Amazon Web Services.
\begin{enumerate}
\item Operating System:Linux(Ubuntu16.04). 
\item IDE: Eclipse (Mars).(3.0)
\item Programming Language : C , javascript
\item API: Web Bluetooth(4.0)
\item AWSIOT Device SDK JS (Version 1.0.12)
\item Google chrome Browser (version :53.0.2785.143)
\end{enumerate}




\chapter{Project Plan}

\section{Project Estimates}
 Our project is based on an Incremental Model. 
 
\subsubsection{Time Estimates}

\begin{table}[!htbp]
\begin{center}
%\def\arraystretch{1.5}
\def\arraystretch{1.5}
\begin{tabularx}{\textwidth}{| X | X |}
\hline
Activity	& Planned months\\
\hline
Requirement gathering and feasibility studying        &1 july – 15 Aug\\
\hline
Planning Activities       &16 Aug – 31 Aug\\
\hline
Designing Modules        &1 sept – 31 Oct\\
\hline
Implementation           &1 Nov – 14 Jan\\
\hline
Testing                  &15 Jan – 15 Feb\\
\hline
Deployment               &16 Feb – 28 Feb\\
\hline



\end{tabularx}
\end{center}
\caption{Project Plan}
\label{tab:usecase}
\end{table}
\newpage
\subsection{Project Resources}
Papers Referred :\\
Zaruba, G.v., S. Basagni, and I. Chlamtac. "Bluetrees-scatternet Formation to Enable Bluetooth-based Ad Hoc Networks." ICC 2001. IEEE International Conference on Communications. Conference Record (Cat. No.01CH37240) (n.d.): n. pag. Web\\ \\
Linthicum, David S. "The Technical Case for Mixing Cloud Computing and Manufacturing." IEEE Cloud Computing 3.4 (2016): 12-15. Web.\\ \\
Massuthe, P., and K. Schmidt. "Operating Guidelines - an Automata-Theoretic Foundation for the Service-Oriented Architecture." Fifth International Conference on Quality Software (QSIC'05) (n.d.): n. pag. Web.\\ \\
Sneps-Sneppe, Manfred, and Dmitry Namiot. "On Physical Web Models." 2016 International Siberian Conference on Control and Communications (SIBCON) (2016): n. pag. Web.\\ \\
Namiot, Dmitry, and Manfred Sneps-Sneppe. "The Physical Web in Smart Cities." 2015 Advances in Wireless and Optical Communications (RTUWO) (2015): n. pag. Web.\\ \\
Lee, Jin-Shyan, Yu-Wei Su, and Chung-Chou Shen. "A Comparative Study of Wireless Protocols: Bluetooth, UWB, ZigBee, and Wi-Fi." IECON 2007 - 33rd Annual Conference of the IEEE Industrial Electronics Society (2007): n. pag. Web.\\ \\
Simeone, Osvaldo, and Haim H. Permuter. "Source Coding with Delayed Side Information." 2012 IEEE International Symposium on Information Theory Proceedings (2012): n. pag. Web.\\ \\
Zhang, Qi, Lu Cheng, and Raouf Boutaba. "Cloud Computing: State-of-the-art and Research Challenges." Journal of Internet Services and Applications 1.1 (2010): 7-18. Web.\\
AWS$-$IOT : https$://$aws.amazon.com$/$documentation$/$iot$/$ \\
Wifi enabled knit board : https$://$github.com$/$Makerville$/$knit\\
\section{Risk Management w.r.t. NP Hard analysis}
Project Risks \\
The dependency on google chrome browser:
The user should have google chrome browser to take advantage of this service as currently the web bluetooth technology is supported by only google chrome.

Fluctuations of Wi-Fi connections.
The vending machine should be placed in the place where wifi availability is there ,for smoother connection and faster product delivery.If the internet fluctuates then user will have to wait for product and this in turn lead to decrease in product sale.
 
Location of the vending machines.
Vending machine should be placed where it can be accesible to the people i.e the bluetooth range .

 
\subsection{Risk Identification}
\begin{enumerate}
\item Have top software and customer managers formally committed to support the project?\\
Yes,the top software company manager has approved our idea and is fully committed to support our project.
\item Are end-users enthusiastically committed to the project and the system/product to be built?\\
The end users in our case being the vendors are happy about the change and betterment we will bring in their bussiness with our platform.
\item Are requirements fully understood by the software engineering team and its customers?\\
The requirements are understood completely and are taken care of by the software engineering team and its customers.
\item Have customers been involved fully in the definition of requirements?\\
The customera are involved and are supporting us for the development of the platform.
\item Do end-users have realistic expectations?\\
 Yes the users do have realistic expectations as our platform will bring a betterment and improve their means of bussiness.
\item Does the software engineering team have the right mix of skills?\\
The software engineering team is the finest we can meet and are at par with their skills.
\item Are project requirements stable?\\
 The project requirements are stable and simple.
\item Is the number of people on the project team adequate to do the job?\\
Yes the number of people on this project are adequate.
\item Do all customer/user constituencies agree on the importance of the project and on the requirements for the system/product to be built?\\
 The customers agree with our idea and are eager to support us in our endeavour.
\end{enumerate}

\subsection{Risk Analysis}
The risks for the Project can be analyzed within the constraints of time and quality

\begin{table}[!htbp]
\begin{center}
%\def\arraystretch{1.5}
\def\arraystretch{1.5}
\begin{tabularx}{\textwidth}{| c | X | c | c | c | c |}
\hline
\multirow{2}{*}{ID} & \multirow{2}{*}{Risk Description}	& \multirow{2}{*}{Probability} & \multicolumn{3}{|c|}{Impact} \\ \cline{4-6}
	& & &	Schedule	& Quality	& Overall \\ \hline
1	& Location of vending machine	& Low	& Low	& High	& High \\ \hline
2	& Availability of WiFi connections	& Low	& Low	& High	& High \\ \hline
\end{tabularx}
\end{center}
\caption{Risk Table}
\label{tab:risk}
\end{table}


\begin{table}[!htbp]
\begin{center}
%\def\arraystretch{1.5}
\def\arraystretch{1.5}
\begin{tabular}{| c | c | c |}
\hline
Probability & Value &	Description \\ \hline
High &	Probability of occurrence is &  $ > 75 \% $ \\ \hline
Medium &	Probability of occurrence is  & $26-75 \% $ \\ \hline
Low	& Probability of occurrence is & $ < 25 \% $ \\ \hline
\end{tabular}
\end{center}
\caption{Risk Probability definitions \cite{bookPressman}}
\label{tab:riskdef}
\end{table}

\begin{table}[!htbp]
\begin{center}
%\def\arraystretch{1.5}
\def\arraystretch{1.5}
\begin{tabularx}{\textwidth}{| c | c | X |}
\hline
Impact & Value	& Description \\ \hline
Very high &	$> 10 \%$ & Schedule impact or Unacceptable quality \\ \hline
High &	$5-10 \%$ & Schedule impact or Some parts of the project have low quality \\ \hline
Medium	& $ < 5 \% $ & Schedule impact or Barely noticeable degradation in quality Low	Impact on schedule or Quality can be incorporated \\ \hline
\end{tabularx}
\end{center}
\caption{Risk Impact definitions \cite{bookPressman}}
\label{tab:riskImpactDef}
\end{table}

\subsection{Overview of Risk Mitigation, Monitoring, Management}


Following are the details for each risk.
\begin{table}[!htbp]
\begin{center}
%\def\arraystretch{1.5}
\def\arraystretch{1.5}
\begin{tabularx}{\textwidth}{| l | X |}
\hline 
Risk ID	& 1 \\ \hline
Risk Description	& Location of the vending machine.\\ \hline
Category	& Development Environment. \\ \hline
Source	& Software requirement Specification document. \\ \hline
Probability	& Low \\ \hline
Impact	& High \\ \hline
Strategy &Do the environment study and then place bluetooth devicefor respective machine \\ \hline
Risk Status	& Identified \\ \hline
\end{tabularx}
\end{center}
%\caption{Risk Impact definitions \cite{bookPressman}}
\label{tab:risk1}
\end{table}

\begin{table}[!htbp]
\begin{center}
%\def\arraystretch{1.5}
\def\arraystretch{1.5}
\begin{tabularx}{\textwidth}{| l | X |}
\hline 
Risk ID	& 2 \\ \hline
Risk Description	& Availability of the WiFi connection.\\ \hline
Category	& Requirements \\ \hline
Source	& Software Design Specification documentation review. \\ \hline
Probability	& Low \\ \hline
Impact	& High \\ \hline
Strategy	& Better testing will resolve this issue.  \\ \hline
Risk Status	& Identified \\ \hline
\end{tabularx}
\end{center}
\label{tab:risk2}
\end{table}
\newpage
\section{Project Schedule}  
\subsection{Project task set}  
Major Tasks in the Project stages are:\\
Task 1: Requirement gathering and feasibility studying.\\
Task 2: Planning Activities.\\
Task 3: Designing Modules\\
Task 4: Implementation. \\
Task 5: Testing.\\
Task 6: Deployment.\\



Task 1:Establishing WiFi connection of the vendng machine.\\
Task 2:Establishing connection between cloud and the vending machine.\\
Task 3:Activation of beacons.\\
Task 4:Find the user location and make online payment available(Mobile wallets). \\
Task 5:Data is updated and stored on the cloud respectively.\\
Task 6:Vendor is informed about the transaction.\\


\subsection{Task network}  
Project tasks and their dependencies are noted in this diagrammatic form.
\begin{center}
	\begin{figure}[!htbp]
		\centering
		\fbox{\includegraphics[height=250pt]{task.png}}
	  \caption{Task Network}
	  \label{fig:act-dig}
	\end{figure}
\end{center}  
\newpage

\subsection{Timeline Chart}  

\begin{center}
	\begin{figure}[!htbp]
		\centering
		\fbox{\includegraphics[height=150pt]{gantt.png}}
	  \caption{Timeline Chart}
	  \label{fig:act-dig}
	\end{figure}
\end{center}  



 
\section{Team Organization}
\subsection{Team structure}
College Guide - Prof A.R.Deshpande\\
Mentor - Mr. Anuj Deshpande \\
Group members -\\
Sejal Khatri\\
Amruta Ranade\\
Kevin Kaul\\
Our project is divided into different smaller modules and the team works independently on  different modules.


\subsection{Management reporting and communication}
We communicate with our college guide and mentor on a regular basis.A log record is maintained which consists of the progress records.
Meetings are conducted with our mentor once every week and the development is recorded.
\chapter{Software reqirement specification  (SRS is to be prepared using relevant mathematics derived and software engg. Indicators in Annex A and B)}

\section{Introduction}
\subsection{Purpose and Scope of Document}
Purpose:\\
An SRS is written in precise, clear and plain language so that it can be reviewed by a business
analyst or customer representative with minimal technical expertise. However it also contains
analytical models (use case diagrams, entity relationship diagrams, data dictionary etc.) which
can be used for the detailed design and the development of the software system. SRS is one of
the most critical pieces of software development since it acts as the bridge betweens the
software developers and business analysts. An incomplete or incorrect SRS can have
disastrous effects on a software project.\\
Scope:\\
Primarily, the scope pertains to the Vending machine providing services to the user and the
vendor. It focuses on the vendor, which allows for the sales, distribution and marketing of the
products through the vending machine.
This SRS is also aimed at specifying requirements of product to be developed but it can also be
applied to assist in the selection of in-house and commercial software products. The standard can
be used to create software requirements specifications directly or can be used as a model for
defining a organization or project specific stan\\
\subsection{Overview of responsibilities of Developer}
The developer will carry out the following activities:
\begin{enumerate}

\item Requirement gathering
\item Planning of the project
\item Designing various modules
\item Implementation of the project
\item Testing of the modules (white box and black box)
\item Deployment of the product (real life usage)

\end{enumerate}
\section{Usage Scenario}
Scenario1: User doesn't get notified :\\
This happens if users bluetooth is not activated ,than he doesn't get the notification .\\
Scenario2:If the wifi connection breaks while the delivery process\\
The details are saved at the cloud side and if the vending machine  is reconnected then it makes the delivery as the payment status is updated .\\\

 \subsection{User profiles}  
There are two actors involved in the use case diagram.\\
1. User\\
2. Vendor\\
The user is the person who approaches the vending machine to buy a product and the vendor is
the person who sells various products by regularly stocking them up in the vending machine. The
relationship between them is similar to a buyer and seller.\\

\subsection{Use Case View}
Use Case Diagram. Example is given below
\begin{center}
	\begin{figure}[!htbp]
		\centering
		\fbox{\includegraphics[width=\textwidth]{usecase.png}}
	  \caption{Use case diagram}
	  \label{fig:usecase}
	\end{figure}
\end{center}  
\newpage
\section{Data Model and Description}  
\subsection{Data objects and Relationships}
  \begin{center}
	\begin{figure}[!htbp]
		\centering
		\fbox{\includegraphics[width=\textwidth]{erd.png}}
	  \caption{erd diagram}
	  \label{fig:usecase}
	\end{figure}
\end{center}  
\newpage
\section{Functional Model and Description}  
Our project consists of these software functions--\\
Interfacing knit board with the vending machine.
In this function we are writing codes and storing them on the stlink programmer.
The stlink works with both its end, one end is connected to the vending machine's motors and the other end is connected to the knit board.
Using the flash process we will store and use the code dynamically.\\

Knit board to AWS (Amazon Web Services)--\\
In this function we will connect the knit board functionalities to the cloud server.
This will be done via the REST(Representational state transfer) protocol.\\

REST : They are one way of providing interoperability between computer systems on the internet. REST-compliant web services allow requesting systems to access and manipulate textual representations of web resources using a uniform and predefined set of stateless operations.  \\
There are six guiding constraints that define a RESTful system.These constraints restrict the ways that the server may process and respond to client requests so that, by operating within these constraints, the service gains desirable non-functional properties, such as performance, scalability, simplicity, modifiability, visibility, portability, and reliability.If a service violates any of the required constraints, it cannot be considered RESTful.\\

Cloud to Device(vendor's side)--\\
In this function we will connect the cloud to the Vendor's device through a mqtt protocol.
Mqtt(MQ Telemetry Transport ) Protocol :\\
It is an ISO standard (ISO/IEC PRF 20922)publish-subscribe-based "lightweight" messaging protocol for use on top of the TCP/IP protocol. It is designed for connections with remote locations where a "small code footprint" is required or the network bandwidth is limited.
Further the vendor will have his own database server where all the data collected will be stored and saved.Here various algorithms will be applied to find the location of the vending machine which needs the reloading of products.
We are providing a platform to the vendors such that they will be able to store the user's past transaction history as well.\\


\begin{center}
	\begin{figure}[!htbp]
		\centering
		\fbox{\includegraphics[height=330pt]{class.png}}
	  \caption{Class diagram}
	  \label{fig:act-dig}
	\end{figure}
\end{center}  
\newpage 
\subsection{Description of functions}  

Fme = Main functions.\\
Fme = (fin , fout,initiate,detect,connect).\\
Fin : {Faddress , Fchioce}\\
Faddress is the function to get user device address and store it in the database .
Fchoice is the function which maps users choice with his particular id ,which further can be used for data analysis \\
Fout:{Fdispose , Fsuggest}\\
Fdispose is the function which is used to validate if payment is done or not and accordingly dispose the product from the vending machine.\\
Fsuggest is the function which suggests product to the user after anlysing its previous choice of products .\\
Finitiate :{Fconnectwifi ,Fflashurl,Fconnectaws}\\
Fconnectwifi :function to connect to the wifi once the device is initiated\\
Fflashurl This function is used to wake up beacon and make it flash url\\
Fconnectaws is used to connect knit board to AWS once its connected to the wifi\\
Fdetect:{Fdetectwifi}\\
Fconnect:{Fcw ,Fcaws}\\
Fcw :refresh connection wifi\\
Fcaws :refresh connection AWS\\
\subsection{Activity Diagram:}

\begin{center}
	\begin{figure}[!htbp]
		\centering
		\fbox{\includegraphics[height=430pt]{activity.png}}
	  \caption{Activity diagram}
	  \label{fig:act-dig}
	\end{figure}
\end{center}  

\newpage

\subsection{Non Functional Requirements:}
Interface Requirements:\\
SPI -Serial Peripheral Interface used to connect Stepper motor driver  to Knit board.\\
Requirement : The driver must support SPI interface and also the board used should support SPI interface .\\
Performance Requirements:\\
Proper Functioning Wifi module to make most of the internet connectivity.\\
Faster AWS responce for quick disposal of products as user would most likely be in hurry in most of the cases\\
Software quality attributes :\\
 Reliability : Complete Reliability on the internet access in the area \\
 Modifiability :Modifiability is supported as the Beacon url is modifiable according to the vendor. \\
 Performance : Performace is measured by day to day testing and respective changes are made to enhance the performance \\
 Security : For security the protocols used are mqtt and https for data transfer.\\
 Testability :Various tests will run at the machine side and updates will be sent on the cloud and respective measures will be taken  \\
 Usability : The will be self adaptable using the testing responce and performance checks,also user adaptability takes place as the data recorded by the user is analysed and respective approach takes place. \\

\subsection{State Diagram:}	
  State Transition Diagram\\

\begin{center}
	\begin{figure}[!htbp]
		\centering
		\fbox{\includegraphics[width=450pt]{state.png}}
	  \caption{State transition diagram}
	  \label{fig:state-dig}
	\end{figure}
\end{center} 
\newpage
 \subsection{Software Interface Description}	 
Software interfacing is done between knit board (wifi enabled board) and Amazon web services  using aws iot js sdk .So the data is updated in time intervals to the cloud .Concept called thing shadows is used at the cloud side to dynamically update data .\\
Interfacing is also done between cloud and vendor device for dynamically publishing the updated data at the AWS .


\chapter{Detailed Design Document using Appendix A and B}
 \section{Introduction}  
 We will have a Wi-Fi enabled board on the vending machines which will send beacons to the mobile phones having Bluetooth in their vicinity.
 
 The user will receive a notification which will contain a URL flashed by the beacon on his cellphone along with suggestions for other products.

The user will hence find and approach the vending machine and click on the desired product on his cellphone.

 The transaction is carried out using the online payment gateways or mobile wallet and is recorded and stored in the cloud which is used for future transactions.

 The vending machine then dispenses the product ordered by the user.

The vendor will have an updated list of the stock remaining in the cloud.  
\section{Architectural Design}  

 
  \begin{center}
	\begin{figure}[!htbp]
		\centering
		\fbox{\includegraphics[width=\textwidth]{Arch.png}}
	  \caption{Architecture diagram}
	  \label{fig:arch-dig}
	\end{figure}
\end{center} 
\newpage

\section{Compoent Design} 
Algorithm used is Dijkatra's algorithm .Dijkstra's algorithm is an algorithm for finding the shortest paths between nodes in a graph, which may represent, for example, road networks.For a given source node in the graph, the algorithm finds the shortest path between that node and every other.[3]:196–206 It can also be used for finding the shortest paths from a single node to a single destination node by stopping the algorithm once the shortest path to the destination node has been determined. For example, if the nodes of the graph represent cities and edge path costs represent driving distances between pairs of cities connected by a direct road, Dijkstra's algorithm can be used to find the shortest route between one city and all other cities. As a result, the shortest path algorithm is widely used in network routing protocols, most notably IS-IS and Open Shortest Path First (OSPF). It is also employed as a subroutine in other algorithms such as Johnson's.\\
Explanation: \\
The cost which we will be using will not just be the cost of the path but the added priority of the product\\
The cost = distance + PriorityMappedValue(depending on product) \\
So the product with highest priority that is which is higher demand  will be given lower value which will reduce the cost and that path will be preferred\\
Accordingly we can calculate and let the vendor know which machine he should approach first\\ 
\subsection{Class Diagram}
 \begin{center}
	\begin{figure}[!htbp]
		\centering
		\fbox{\includegraphics[width=450pt]{class.png}}
	  \caption{Class Diagram}
	  \label{fig:class-dig}
	\end{figure}
\end{center} 
 

\begin{appendices}
\begin{enumerate}
\section{References}
\item Zaruba, G.v., S. Basagni, and I. Chlamtac. "Bluetrees-scatternet Formation to Enable Bluetooth-based Ad Hoc Networks." ICC 2001. IEEE International Conference on Communications. Conference Record (Cat. No.01CH37240) (n.d.): n. pag. Web\\
\item Linthicum, David S. "The Technical Case for Mixing Cloud Computing and Manufacturing." IEEE Cloud Computing 3.4 (2016): 12-15. Web.\\
\item Massuthe, P., and K. Schmidt. "Operating Guidelines - an Automata-Theoretic Foundation for the Service-Oriented Architecture." Fifth International Conference on Quality Software (QSIC'05) (n.d.): n. pag. Web.\\
\item Sneps-Sneppe, Manfred, and Dmitry Namiot. "On Physical Web Models." 2016 International Siberian Conference on Control and Communications (SIBCON) (2016): n. pag. Web.\\
\item Namiot, Dmitry, and Manfred Sneps-Sneppe. "The Physical Web in Smart Cities." 2015 Advances in Wireless and Optical Communications (RTUWO) (2015): n. pag. Web.\\
\item Lee, Jin-Shyan, Yu-Wei Su, and Chung-Chou Shen. "A Comparative Study of Wireless Protocols: Bluetooth, UWB, ZigBee, and Wi-Fi." IECON 2007 - 33rd Annual Conference of the IEEE Industrial Electronics Society (2007): n. pag. Web.\\
\item Simeone, Osvaldo, and Haim H. Permuter. "Source Coding with Delayed Side Information." 2012 IEEE International Symposium on Information Theory Proceedings (2012): n. pag. Web.\\
\item Zhang, Qi, Lu Cheng, and Raouf Boutaba. "Cloud Computing: State-of-the-art and Research Challenges." Journal of Internet Services and Applications 1.1 (2010): 7-18. Web.\\


\end{enumerate}   

\section{Summary and Conclusion}
We expect to learn web bluetooth technology and its applications .Also to learn how a actual product is launched and how to take care of the finished product delivery.
Steps in beginning of  this project from collecting the vendor requirements to take  care of the user test cases involved and then simulatng the environment to predict future risks and then applying Risk Management on it .


% \chapter{ALGORITHMIC DESIGN}
\chapter{Laboratory assignments on Project Analysis of Algorithmic Design}

To develop the problem under consideration and justify feasibilty using
concepts of knowledge canvas and IDEA Matrix.\\
Refer \cite{innovationbook} for IDEA Matrix and Knowledge canvas model. Case studies are given in this book. IDEA Matrix is represented in the following form. Knowledge canvas represents about identification  of opportunity for product. Feasibility is represented w.r.t. business perspective.\\ 

\begin{table}[!htbp]
\begin{center}
  \begin{tabular}{| c | c | c | c |}
\hline
 I & D & E & A \\ 
\hline
Increase & Drive & Educate & Accelerate \\
\hline
Improve & Deliver & Evaluate & Associate  \\
 \hline
Ignore & Decrease & Eliminate & Avoid \\
\hline
\end{tabular}
 \caption { IDEA Matrix }
 \label{tab:imatrix}
\end{center}
\end{table}

Project problem statement feasibility assessment using NP-Hard, NP-Complete or satisfy ability issues using modern algebra and/or relevant mathematical models.
 input x,output y, y=f(x)











\chapter{Laboratory assignments on Project Quality and Reliability Testing of Project Design}

It should include assignments such as\\
 Use of divide and conquer strategies to exploit distributed/parallel/concurrent processing of the above to identify object, morphisms, overloading in functions (if any), and functional relations and any other dependencies (as per requirements).\\
             It can include Venn diagram, state diagram, function relations, i/o relations; use this to derive objects, morphism, overloading\\

 Use of above to draw functional dependency graphs and relevant Software modeling methods, techniques including UML diagrams or other necessities using appropriate tools.
 \\
Testing of project problem statement using generated test data (using mathematical models, GUI, Function testing principles, if any) selection and appropriate use of testing tools, testing of UML diagram's reliability. Write also test cases [Black box testing] for each identified functions. \\
 Additional assignments by the guide. If project type as Entreprenaur, Refer \cite{ehr},\cite{mckinsey},\cite{mckinseyweb}, \cite{govwebsite}


\chapter{Project Planner}
\label{app:plan}
Using planner or alike project management tool.




\chapter{Reviewers Comments of Paper Submitted}

\begin{enumerate}
\item Paper Title: Physical Web with Vending Machine

\item Name of the Conference/Journal where paper can be  submitted :
 Physical Web in Smart Cities - Advances in Wireless and Optical Communications (RTUWO), 2015\\
 On physical web models - Control and Communications (SIBCON), 2016 \\
International Siberian Conference Finite state machine based vending machine – International Journal of VLSI design and communication system 2012.\\


\item Paper accepted/rejected : Not applied yet. 
\item Review comments by reviewer : not yet applied 
\item Corrective actions if any :   not yet applied 

\end{enumerate}

\chapter{Plagiarism Report}
Plagiarism report


\end{appendices}


\end{document}
